
%% bare_conf.tex
%% V1.4
%% 2012/12/27
%% by Michael Shell
%% See:
%% http://www.michaelshell.org/
%% for current contact information.
%%
%% This is a skeleton file demonstrating the use of IEEEtran.cls
%% (requires IEEEtran.cls version 1.8 or later) with an IEEE conference paper.
%%
%% Support sites:
%% http://www.michaelshell.org/tex/ieeetran/
%% http://www.ctan.org/tex-archive/macros/latex/contrib/IEEEtran/
%% and
%% http://www.ieee.org/

%%*************************************************************************
%% Legal Notice:
%% This code is offered as-is without any warranty either expressed or
%% implied; without even the implied warranty of MERCHANTABILITY or
%% FITNESS FOR A PARTICULAR PURPOSE! 
%% User assumes all risk.
%% In no event shall IEEE or any contributor to this code be liable for
%% any damages or losses, including, but not limited to, incidental,
%% consequential, or any other damages, resulting from the use or misuse
%% of any information contained here.
%%
%% All comments are the opinions of their respective authors and are not
%% necessarily endorsed by the IEEE.
%%
%% This work is distributed under the LaTeX Project Public License (LPPL)
%% ( http://www.latex-project.org/ ) version 1.3, and may be freely used,
%% distributed and modified. A copy of the LPPL, version 1.3, is included
%% in the base LaTeX documentation of all distributions of LaTeX released
%% 2003/12/01 or later.
%% Retain all contribution notices and credits.
%% ** Modified files should be clearly indicated as such, including  **
%% ** renaming them and changing author support contact information. **
%%
%% File list of work: IEEEtran.cls, IEEEtran_HOWTO.pdf, bare_adv.tex,
%%                    bare_conf.tex, bare_jrnl.tex, bare_jrnl_compsoc.tex,
%%                    bare_jrnl_transmag.tex
%%*************************************************************************


% Note that the a4paper option is mainly intended so that authors in
% countries using A4 can easily print to A4 and see how their papers will
% look in print - the typesetting of the document will not typically be
% affected with changes in paper size (but the bottom and side margins will).
% Use the testflow package mentioned above to verify correct handling of
% both paper sizes by the user's LaTeX system.
%
% Also note that the "draftcls" or "draftclsnofoot", not "draft", option
% should be used if it is desired that the figures are to be displayed in
% draft mode.
%
\documentclass[conference]{IEEEtran}
% Add the compsoc option for Computer Society conferences.
%
% If IEEEtran.cls has not been installed into the LaTeX system files,
% manually specify the path to it like:
% \documentclass[conference]{../sty/IEEEtran}





% Some very useful LaTeX packages include:
% (uncomment the ones you want to load)


% *** MISC UTILITY PACKAGES ***

% *** CITATION PACKAGES ***
%
\usepackage{cite}
% cite.sty was written by Donald Arseneau
% V1.6 and later of IEEEtran pre-defines the format of the cite.sty package
% \cite{} output to follow that of IEEE. Loading the cite package will
% result in citation numbers being automatically sorted and properly
% "compressed/ranged". e.g., [1], [9], [2], [7], [5], [6] without using
% cite.sty will become [1], [2], [5]--[7], [9] using cite.sty. cite.sty's
% \cite will automatically add leading space, if needed. Use cite.sty's
% noadjust option (cite.sty V3.8 and later) if you want to turn this off
% such as if a citation ever needs to be enclosed in parenthesis.
% cite.sty is already installed on most LaTeX systems. Be sure and use
% version 4.0 (2003-05-27) and later if using hyperref.sty. cite.sty does
% not currently provide for hyperlinked citations.
% The latest version can be obtained at:
% http://www.ctan.org/tex-archive/macros/latex/contrib/cite/
% The documentation is contained in the cite.sty file itself.






% *** GRAPHICS RELATED PACKAGES ***
%
\ifCLASSINFOpdf
  \usepackage[pdftex]{graphicx}
  % declare the path(s) where your graphic files are
  % \graphicspath{{../pdf/}{../jpeg/}}
  % and their extensions so you won't have to specify these with
  % every instance of \includegraphics
  % \DeclareGraphicsExtensions{.pdf,.jpeg,.png}
\else
  % or other class option (dvipsone, dvipdf, if not using dvips). graphicx
  % will default to the driver specified in the system graphics.cfg if no
  % driver is specified.
  \usepackage[dvips]{graphicx}
  % declare the path(s) where your graphic files are
  % \graphicspath{{../eps/}}
  % and their extensions so you won't have to specify these with
  % every instance of \includegraphics
  % \DeclareGraphicsExtensions{.eps}
\fi
% graphicx was written by David Carlisle and Sebastian Rahtz. It is
% required if you want graphics, photos, etc. graphicx.sty is already
% installed on most LaTeX systems. The latest version and documentation
% can be obtained at: 
% http://www.ctan.org/tex-archive/macros/latex/required/graphics/
% Another good source of documentation is "Using Imported Graphics in
% LaTeX2e" by Keith Reckdahl which can be found at:
% http://www.ctan.org/tex-archive/info/epslatex/
%
% latex, and pdflatex in dvi mode, support graphics in encapsulated
% postscript (.eps) format. pdflatex in pdf mode supports graphics
% in .pdf, .jpeg, .png and .mps (metapost) formats. Users should ensure
% that all non-photo figures use a vector format (.eps, .pdf, .mps) and
% not a bitmapped formats (.jpeg, .png). IEEE frowns on bitmapped formats
% which can result in "jaggedy"/blurry rendering of lines and letters as
% well as large increases in file sizes.
%
% You can find documentation about the pdfTeX application at:
% http://www.tug.org/applications/pdftex



% correct bad hyphenation here
\hyphenation{op-tical net-works semi-conduc-tor}
\usepackage[utf8]{inputenc}
\usepackage[T1]{fontenc}


\begin{document}
%
% paper title
% can use linebreaks \\ within to get better formatting as desired
% Do not put math or special symbols in the title.
\title{Türkçe Haber Metinleri Üzerinden Popülerlik Tahmini}


% author names and affiliations
% use a multiple column layout for up to three different
% affiliations
\author{\IEEEauthorblockN{Eran Toker}
\IEEEauthorblockA{Hacettepe Üniversitesi\\
Ankara, Türkiye\\
Email: erantoker@gmail.com}
\and
\IEEEauthorblockN{Fatih Güler}
\IEEEauthorblockA{Hacettepe Üniversitesi\\
Ankara, Türkiye\\
Email: ffguler@gmail.com}
\and
\IEEEauthorblockN{Yiğit Sever}
\IEEEauthorblockA{Hacettepe Üniversitesi\\
Ankara, Türkiye\\
Email: yigitsever94@gmail.com}}


% make the title area
\maketitle

% As a general rule, do not put math, special symbols or citations
% in the abstract
\begin{abstract}
Haber kaynakları her gün birçok haber makalesi yayınlamakta ancak yayınlanan bu metinlerin oldukça küçük bir bölümü okuyucuların ilgisini çekip popüler olabilmektedir. Popülerlik bir makalenin aldığı tıklanma sayısı aracılığıyla ölçülebilse de kullanıcıların ilgi duydukları haberleri sosyal medya aracılığıyla paylaşıp bir nevi bu metinlerin popülerliği konusunda oy kullanabildikleri fikri göz önüne alındığında popülerliği ölçmek için yenilikçi yöntemler kullanılabileceği anlaşılır. Bu çalışmada, Milliyet.com.tr üzerinden paylaşılan haberlerin, Türkçe sosyal medya ve tartışma platformu EkşiSözlük.com üzerinden paylaşılıp paylaşılmayacağını tahmin etmeye çalışıyoruz.
\end{abstract}

\section{Introduction}
% no \IEEEPARstart

The advancement and steady growth in social media, mainly blogosphere and streaming sites, allowed new types of data to become available which can be mined for valuable knowledge. To give an example; online discussions can be used to predict sales ranks of books.\cite{gruhl_predictive_2005} Copious amounts of posts on social media are posted as a response to events that users read from news articles, as a result, investigating events and their social impact that is reflected in social media has become an important task for media analysts.

Our purpose in this work is predicting the popularity of a news article before it goes live. Predicting popularity and traffic of an article might help with determining the desirability of an article, whether or not it is worth to publish and advertise on it. \cite{phukan_feasibility_2016} If news agents knew which articles can or will be popular, they can spend their resources on potential candidates. This concept is also important in the sense of serving a better user experience; news sources can offer articles which might be more interesting to readers. If online news agencies can gain more users, they can earn more money from ad revenue.\cite{bandari_pulse_2012}

Our aim in this paper is to predict popularity of news articles before publication. Popularity is used in the sense that whether or not the article will be shared on EksiSozluk.com which is the most popular social media establishment in Turkey, with Turkish origins.

This work makes several contributions. First, it explores the dynamics of stories in  Milliyet.com.tr. Second, it introduces the problem of predicting the popularity of a news article. Third, it provides a set of textual and semantic features that can be used to predict if a news article will be shared or not in EksiSozluk.com before the said article is published. Fourth, it provides an evaluation of the introduced features. Fifth, an error analysis identifies possible causes for classification failure.

 
\section{Related Work}
Most of the popularity prediction works based their approach on early comments/likes/retweets. \cite{szabo_predicting_2010} The implication is, the content actually needs to be published, losing valuable aspects that will be gained from a prior analysis.
In a research conducted in 2009 \cite{tsagkias_predicting_2009}, they have used a 5 feature set for finding popularity; surface, cumulative, textual, semantic and real world. Surface feature consisted of images present in the article or how many authors have contributed to the article. Cumulative feature looked for how many articles were published at the same hour. Real world feature analyzed whether or not the weather was nice or cloudy. Subsequently, we used \emph{textual} and \emph{semantic} features since those are the only features that are related to text mining.

\section{Data and Features}
\subsection{Exploring News}
The data we used in the project consists of 449 news stories published at Milliyet.com.tr between February 2016 to August 2016. Among those 449 news stories, 60 of them are tagged as 'unpopular' by us since they were not shared in Eksisozluk.com. The reason for relatively small sample size is that there were no set methods or data sets of Turkish news articles that are linked to threads opened at Eksisozluk.com, as a result, the dataset was put together by us via hand. Another challenge was the fact that Eksisozluk.com is not a social media that users share content with links and have discussion based on them (i.e. Twitter, Reddit) but a discussion forum with topics and thread titles decided by users. Therefore, an automated method would not be plausible.

The selected news articles were then parsed and their content extracted. News articles come with a heading, a subheading and the actual text content which we appended together and handled as news body.

Our first research question is, what are the dynamics of user generated comments on news articles? According to Tsagkias et al. \cite{tsagkias_predicting_2009}, there were 5 features with two of them about text mining that we used. Textual feature means getting terms at top 200 frequency in our corpus. Terms were ranked by their log-likelihood scores for all stories. Semantic feature is the count of people, location and organization mentioned in the articles.

\subsection{Feature Engineering}
For textual feature, we first found most common 200 terms in all stories. We skipped stop words in Turkish and used Resha Turkish Stemmer \cite{resha-turkish-stemmer} for stemming operations. As we initially predicted; "Basbakan", "Cumhurbaskani", "Futbol", "Polis", "Asker", "Sehit", "Patlama", "Teror" tokens were some of the most common terms as those were related to common occurrences in Turkey, 2016.

For semantic feature, first we tried using ITU Natural Language Processing Toolkit. With this tool, we were able to find Named Entities of each word in the articles. According to pipeline we used the schema represented in Figure \ref{fig_sim}. Thus we were able to get person, organization and location counts for each news article. However, this tool was unfeasibly slow; a query for one single word took more than a minute at times. Therefore, we switched to DBPedia-Spotlight \cite{isem2013daiber}. This tool offered in-house querying instead of API querying and we could get person, organization and location results. After that we used Random Forest Algorithm \cite{random-forest}. We had 4 numerical data as in Term Frequency of popular terms, Person Count, Organization Count and Location Count as well as one label which is popular or not. Consequently, we reimplemented GitHub user 'ironmanMa's algorithm for our purpose.
\begin{figure}[!t]
	\includegraphics[width=0.8\textwidth,natwidth=800,natheight=80]{schema.png}
	% where an .eps filename suffix will be assumed under latex, 
	% and a .pdf suffix will be assumed for pdflatex; or what has been declared
	% via \DeclareGraphicsExtensions.
	\caption{Pipeline Schema.}
	\label{fig_sim}
\end{figure}
\section{Experiment Setup}
As mentioned beforehand, EksiSozluk.com does not provide an API, so we manually found news stories shared in EksiSozluk.com. After creating a heap of stories that have been talked about in EksiSozluk.com, we extracted the news stories from webpages and categorized them by month. Then, most common 200 terms were found and written to 'MostCommonTerms' file. Later, we processed input file for textual features and wrote to 'TermFrequencies' file. Using DBpedia Spotlight, threads wrote to output file 'SemanticVariables' fir each month. Via 3 output files, we have created 'TreeData' files for each test and training stories and wrote to 'TreeData' and 'TestTreeData' file. Using tree data files we used random forest implementation for results. Then we have calculated F1 scores and accuracy.

\section{Results}

The results looked promising as shown in Table \ref{prf1_lc}, approaching results achieved by Bandari et al. \cite{bandari_pulse_2012} with precision score of 0.8.


%%%%%%%%%%%%	LOWER CASE TEST RESULTS		%%%%%%%%%%%%
\begin{table}[!t]
	\caption{Test Results for Lowercase dataset}
	\label{test_result_lc}
	\centering
	\begin{tabular}{|c|c|}
		\hline 
		True Positive & 33 \\ 
		\hline 
		False Positive & 8 \\ 
		\hline 
		False Negative & 15 \\ 
		\hline 
	\end{tabular} 
\end{table}
%%%%%%%%%%%%	LOWER CASE TEST RESULTS		%%%%%%%%%%%%
%%%%%%%%%%%%	LOWER CASE P R F1		%%%%%%%%%%%%
\begin{table}[!t]
	\caption{Precision, Recall and F1 Scores for Lowercase dataset}
	\label{prf1_lc}
	\centering
	\begin{tabular}{|c|c|}
		\hline
		Precision & 0.80 \\
		\hline
		Recall & 0.69 \\
		\hline
		F1 Score & 0.74 \\
		\hline
	\end{tabular}
\end{table}

%%%%%%%%%%%%	LOWER CASE P R F1		%%%%%%%%%%%%

 As mentioned before, we opted to convert our dataset to lowercase for the experiments. However, this lead DBpedia Spotlight tool to miss person information, while not having an impact on location or organization information. To eliminate this error, we reverted our dataset back to sentence case (unaltered case) and ran the tests again. Results are shown in Table \ref{test_result_sc} and \ref{prf1_sc} and are improved from lowercase dataset tests by a noticeable margin.
%%%%%%%%%%%%	Sentence CASE TEST RESULTS		%%%%%%%%%%%%
\begin{table}[!t]
	\caption{Test Results for Sentence Case dataset}
	\label{test_result_sc}
	\centering
	\begin{tabular}{|c|c|}
		\hline 
		True Positive & 36 \\ 
		\hline 
		False Positive & 7 \\ 
		\hline 
		False Negative & 12 \\ 
		\hline 
	\end{tabular} 
\end{table}
%%%%%%%%%%%%	Sentence CASE TEST RESULTS		%%%%%%%%%%%%
%%%%%%%%%%%%	Sentence CASE P R F1		%%%%%%%%%%%%
\begin{table}[!t]
	\caption{Precision, Recall and F1 Scores for Sentence dataset}
	\label{prf1_sc}
	\centering
	\begin{tabular}{|c|c|}
		\hline
		Precision & 0.83 \\
		\hline
		Recall & 0.75 \\
		\hline
		F1 Score & 0.79 \\
		\hline
	\end{tabular}
\end{table}

%%%%%%%%%%%%	Sentence CASE TEST RESULTS		%%%%%%%%%%%%

\section{Conclusion}
In this work, we aimed to predict the popularity of news articles prior to their publication. We have created a dataset and picked our features regarding earlier works in the literature. After the evaluation steps, we have looked into the misclassified stories. Most of the stories that have been labeled as 'popular' by our implementation but do not have a EksiSozluk.com thread associated to it are actually popular but just not by our parameters. These false positives did have discussions about them, but on threads of a popular person mentioned in the article. We did not have an autonomous method of finding these links. Another case of misclassified stories included a lot of famous people or organizations but the stories themselves were not interesting or worth talking about (i.e. clickbait).


% An example of a floating figure using the graphicx package.
% Note that \label must occur AFTER (or within) \caption.
% For figures, \caption should occur after the \includegraphics.
% Note that IEEEtran v1.7 and later has special internal code that
% is designed to preserve the operation of \label within \caption
% even when the captionsoff option is in effect. However, because
% of issues like this, it may be the safest practice to put all your
% \label just after \caption rather than within \caption{}.
%
% Reminder: the "draftcls" or "draftclsnofoot", not "draft", class
% option should be used if it is desired that the figures are to be
% displayed while in draft mode.
%


% Note that IEEE typically puts floats only at the top, even when this
% results in a large percentage of a column being occupied by floats.


% An example of a double column floating figure using two subfigures.
% (The subfig.sty package must be loaded for this to work.)
% The subfigure \label commands are set within each subfloat command,
% and the \label for the overall figure must come after \caption.
% \hfil is used as a separator to get equal spacing.
% Watch out that the combined width of all the subfigures on a 
% line do not exceed the text width or a line break will occur.
%
%\begin{figure*}[!t]
%\centering
%\subfloat[Case I]{\includegraphics[width=2.5in]{box}%
%\label{fig_first_case}}
%\hfil
%\subfloat[Case II]{\includegraphics[width=2.5in]{box}%
%\label{fig_second_case}}
%\caption{Simulation results.}
%\label{fig_sim}
%\end{figure*}
%
% Note that often IEEE papers with subfigures do not employ subfigure
% captions (using the optional argument to \subfloat[]), but instead will
% reference/describe all of them (a), (b), etc., within the main caption.


% An example of a floating table. Note that, for IEEE style tables, the 
% \caption command should come BEFORE the table. Table text will default to
% \footnotesize as IEEE normally uses this smaller font for tables.
% The \label must come after \caption as always.
%
%\begin{table}[!t]
%%% increase table row spacing, adjust to taste
%%\renewcommand{\arraystretch}{1.3}
%% if using array.sty, it might be a good idea to tweak the value of
%% \extrarowheight as needed to properly center the text within the cells
%\caption{An Example of a Table}
%\label{table_example}
%\centering
%%% Some packages, such as MDW tools, offer better commands for making tables
%%% than the plain LaTeX2e tabular which is used here.
%\begin{tabular}{|c||c|}
%\hline
%One & Two\\
%\hline
%Three & Four\\
%\hline
%\end{tabular}
%\end{table}

% Note that IEEE does not put floats in the very first column - or typically
% anywhere on the first page for that matter. Also, in-text middle ("here")
% positioning is not used. Most IEEE journals/conferences use top floats
% exclusively. Note that, LaTeX2e, unlike IEEE journals/conferences, places
% footnotes above bottom floats. This can be corrected via the \fnbelowfloat
% command of the stfloats package.






% conference papers do not normally have an appendix


% use section* for acknowledgement
%\section*{Acknowledgment}


%The authors would like to thank...


% references section

% can use a bibliography generated by BibTeX as a .bbl file
% BibTeX documentation can be easily obtained at:
% http://www.ctan.org/tex-archive/biblio/bibtex/contrib/doc/
% The IEEEtran BibTeX style support page is at:
% http://www.michaelshell.org/tex/ieeetran/bibtex/
\bibliographystyle{IEEEtran}
% argument is your BibTeX string definitions and bibliography database(s)
\bibliography{bibi}
%
% <OR> manually copy in the resultant .bbl file
% set second argument of \begin to the number of references
% (used to reserve space for the reference number labels box)
%\begin{thebibliography}{2}
%
%\bibitem{IEEEhowto:kopka}
%H.~Kopka and P.~W. Daly, \emph{A Guide to \LaTeX}, 3rd~ed.\hskip 1em plus
%  0.5em minus 0.4em\relax Harlow, England: Addison-Wesley, 1999.
%
%\end{thebibliography}




% that's all folks
\end{document}


